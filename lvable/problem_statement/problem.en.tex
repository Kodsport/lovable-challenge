\problemname{``lv''-able}
%% plainproblemname: "lv"-able
\noindent
A string is ``\texttt{lv}''-able if it contains the contiguous substring ``\texttt{lv}''.

You are given a string $s$ with $N$ characters, and you want to make it ``\texttt{lv}''-able in as few operations as possible.

You are allowed to do any of these operations:
\begin{itemize}
  \item Remove any character at any position.
  \item Insert any character at any position.
  \item Replace any character by any other character at any position.
  \item Choose any consecutive interval of characaters, and reverse it the order of the characters in it.
\end{itemize}

Now make the string ``\texttt{lv}''-able!

\section*{Input}
The first line contains an integer $N$ ($1 \leq N \leq 5 \cdot 10^5$), the number of characaters in the initial string.

The second line contains a string with $N$ characters, containing only of lowercase letters \texttt{a}-\texttt{z}.

\section*{Output}
Print an integer: the minimum number operations such that the string $s$ becomes ``\texttt{lv}''-able.

\section*{Scoring}
Your solution will be tested on a set of test groups, each worth a number of points. Each test group contains
a set of test cases. To get the points for a test group you need to solve all test cases in the test group.

\noindent
\begin{tabular}{| l | l | p{12cm} |}
  \hline
  \textbf{Group} & \textbf{Points} & \textbf{Constraints} \\ \hline
  $1$    & $40$       & $N \leq 100$ \\ \hline
  $2$    & $60$       & No additional constraints. \\ \hline
\end{tabular}

\section*{Explanation of Samples}
In sample $1$, you can reverse the substring ``\texttt{ov}'', resulting in \texttt{lvoable}, which then contains
``\texttt{lv}''.

In sample $2$, we can replace the ``\texttt{e}'' with a ``\texttt{v}'', which then contains ``\texttt{lv}''.

In sample $3$, the string already contains ``\texttt{lv}''.
