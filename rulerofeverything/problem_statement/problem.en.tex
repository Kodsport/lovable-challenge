\problemname{Ruler of Everything}

\noindent
You have suddenly developed egomaniacal tendencies and desire fame. You want to get $K$ followers or more on the video platform
Dutub. To do this, you have planned $N$ video ideas. Thanks to your outstanding intellectual abilities, you know exactly
how releasing each one will affect your following. Each video has two parameters, $a$ and $b$. If you currently have $f$ followers
and release a video, your amount of followers afterwards will be $a \cdot f + b$. 

To attain your goal as soon as possible, you now want to release the smallest number of videos to obtain at least $K$ followers.
You can release your chosen videos in any order, but you can only release each video idea at most once.

Because you are impatient, you might change your mind about $K$. In fact, you have to find the minimum number of videos
for $Q$ different values of $K$.

\section*{Input}
The first line of input contains the integers $N$ and $Q$ ($1 \le N \le 2 \cdot 10^5$), the number video ideas
and how follower amounts you want to consider.

The following each lines describe your video ideas. Each line contains two integers $a$ and $b$ ($1 \leq a, b \leq 10^5$),
whose meaning is explained above.

The final line contains $Q$ integers $K_1, K_2, \dots, K_N$ ($1 \leq K_i \leq 8 \cdot 10^9$), the different follow counts
you want to consider.

\section*{Output}
For each $K_i$, print the smallest number of videos you need to release to get $K_i$ followers.
Print all of these answers on a single line.

\section*{Scoring}
Your solution will be tested on a set of test groups, each worth a number of points. Each test group contains
a set of test cases. To get the points for a test group you need to solve all test cases in the test group.

\noindent
\begin{tabular}{| l | l | p{12cm} |}
  \hline
  \textbf{Group} & \textbf{Points} & \textbf{Constraints} \\ \hline
  $1$    & $2$        & $Q \leq 5, N \leq 10$ \\ \hline
  $2$    & $13$       & $Q \leq 5, N \leq 20$ \\ \hline
  $3$    & $20$       & $Q \leq 5, N \leq 1000$ \\ \hline
  $4$    & $25$       & $Q \leq 5$ \\ \hline
  $5$    & $20$       & $Q \leq 1000$ \\ \hline
  $6$    & $20$       & No additional constraints. \\ \hline
\end{tabular}

\section*{Explanation of Sample 1}
